\documentclass[../Article_Sensitivity_Analsysis.tex]{subfiles}
\graphicspath{{\subfix{../Figures/}}}
\begin{document}
	
	This study aims to analyze the influence of changes in operating conditions on the supercritical extraction process described in {\color{red}article 1}. The emphasis is put on the effect of the mass flow rate, pressure, and the inlet temperature. The relation between input and output is obtained by applying a sensitivity analysis. Sensitivity analysis examines the impact of varying inputs or model parameters on the system's output. The aim is to understand and to allocate the source of uncertainty in the output to the corresponding inputs or parameters. There are many sensitivity analysis methods, which include but are not limited to those listed below:
	
	\begin{itemize}
		\item One-at-a-time method
		\item Derivative-based local methods
		\item Variance-based methods
	\end{itemize}
	
	Different supercritical extraction models were analyzed using sensitivity analysis. \citet{Fiori_2007}, performed the sensitivity calculations by varying the parameters within their confidence interval and observing how the model results changed. This allows to evaluate the effect of the uncertainties on model predictions. The sensitivity analysis revealed that the particle diameter and the internal mass transfer coefficient are significant for the extraction process. The effect of changing some operative conditions was also investigated, underlining how the solvent flow rate and the seed milling affect the extraction process.
	
	\citet{Santos2000}, in their work, considered the model of \citet{Sovova1994} for the semi-continuous isothermal and isobaric extraction processes using carbon dioxide as a solvent. The parametric sensitivity analysis using a factorial design in two levels was carried out. The model parameters were disturbed by 10\%, and their main effects were analysed so that it is possible to propose strategies for high-performance operation. The authors performed the sensitivity analysis with respect to the superficial velocity, particle diameter, initial concentration of solute in the solid phase and the concentration of solute in the fluid phase at the inlet to the extractor.
	
	\citet{Hatami2024} performed a one-factor-at-a-time sensitivity analysis to	assess the response of net present value concerning variations in both technical and economic variables. The analysis is divided into two parts. In the first part, the sensitivity analysis is carried out for a SFE plant with a constant	extractor volume of 300 L. Here, the focus is on examining how net present value is influenced by changes in individual technical and economic parameters while keeping the extractor volume fixed. In the second part, the investigation revolves around the effects of altering the extractor volume (between 1 to 600 L) on the overall profitability of the project. Their findings show that the most influential factors on NPV include the price of the extract, the interest rate, the dynamic time of SFE, and the project lifetime.
	
	\citet{Poletto1996} provided a general dimensionless model for the supercritical extraction process of vegetable and essential oils and applied a sensitivity analysis. The sensitivity calculations were performed by varying the parameters and analysing the model response. The authors found that a dimensionless partition coefficient and a dimensionless characteristic time appeared as the most important parameters of the extraction process. 
	
\end{document}