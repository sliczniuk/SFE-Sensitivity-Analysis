\documentclass[../Article_Sensitivity_Analsysis.tex]{subfiles}
\graphicspath{{\subfix{../Figures/}}}
\begin{document}
	
	
	This study aims to analyze the influence of changes in operating conditions on the process model described by \citet{Sliczniuk2024}. Sensitivity analysis is applied to examine the impact of model parameters or controls on the model output. The results of sensitivity analysis can be used to identify sources of uncertainty, simplify the model, or detect errors by revealing unexpected relationships between inputs and outputs. Various sensitivity analysis methods are available, including:
	
	\begin{itemize}
		\item One-at-a-time method
		\item Derivative-based local methods
		\item Variance-based methods
	\end{itemize}
	
	Different supercritical extraction models have been analyzed using various sensitivity analysis techniques in the literature. For instance, \citet{Fiori_2007} performed sensitivity calculations by varying parameters within their confidence intervals and observing the changes in model results. Their analysis revealed that particle diameter and internal mass transfer coefficient significantly influence extraction during the diffusion-control regime.
	
	\citet{Santos2000} considered the model of \citet{Sovova1994} for semi-continuous isothermal and isobaric extraction processes using carbon dioxide as a solvent. They conducted a parametric sensitivity analysis using a two-level factorial design, disturbing model parameters by 10\% and analyzing their main effects. They proposed strategies for high-performance operation based on sensitivities related to superficial velocity, particle diameter, initial solute concentration in the solid phase, and solute concentration in the fluid phase at the extractor inlet.
	
	\citet{Hatami2024} performed a one-factor-at-a-time sensitivity analysis to assess the response of net present value to variations in technical and economic variables. Their study consists of two parts. The first part examines how net present value is influenced by changes in individual technical and economic parameters, keeping the extractor volume constant at 300 L. The second part investigates the effects of varying the extractor volume (from 1 to 600 L) on the project's overall profitability. {\color{blue}The authors found that the raw material price, discount rate and residence time had the biggest impact on NPV}.
	%Their findings indicate that the most influential factors on net present value include the price of the extract, interest rate, dynamic time, and project lifetime.
	
	%\citet{Poletto1996} provided a general dimensionless model for the supercritical extraction process of vegetable and essential oils. The sensitivity calculations were performed by varying the parameters and analysing the model response. The authors found that a dimensionless partition coefficient and a dimensionless characteristic time appeared as the most important parameters of the extraction process. 
	
\end{document}