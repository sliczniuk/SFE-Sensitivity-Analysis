\documentclass[../Article_Sensitivity_Analsysis.tex]{subfiles}
\graphicspath{{\subfix{../Figures/}}}
\begin{document}
	
	This study aims to analyse the influence of changes in operating conditions on the supercritical extraction model described in {\color{red}article 1}. The relation between input and output is obtained by applying a sensitivity analysis. Sensitivity analysis examines the impact of varying inputs or model parameters on the system's output. The aim is to understand and allocate the source of uncertainty in the output to the corresponding inputs or parameters. There are many sensitivity analysis methods, which include but are not limited to those listed below:
	
	\begin{itemize}
		\item One-at-a-time method
		\item Derivative-based local methods
		\item Variance-based methods
	\end{itemize}
	
	Different supercritical extraction models were analysed using sensitivity analysis. \citet{Fiori_2007}, performed the sensitivity calculations by varying the parameters within their confidence interval and observing how the model results changed. This allows to evaluate the effect of the uncertainties on model predictions. The sensitivity analysis revealed that the particle diameter and the internal mass transfer coefficient are the most influential factors in the extraction process. The effect of changing operating conditions was also investigated, underlining how solvent flow rate and seed milling affect extraction.
	
	In the work of \citet{Santos2000}, a diffusive model for the semi-continuous, isothermal, and isobaric extraction process is analysed by a parametric sensitivity analysis using a factorial design in two levels. The parametric sensitivity analysis was carried out by applying disturbances of 10\% in the values of nominal values of velocity, particle diameter, and initial concentration in the solid and liquid phases. The authors concluded that it is necessary to determine the optimum particle diameter and manipulate the solvent superficial velocity to control the process.
	
	\citet{Hatami2024} used a one-factor-at-a-time sensitivity analysis and employed two distinct strategies to conduct their analysis: the first strategy involved applying an equal fluctuation to all technical and economic parameters, while the second strategy utilised real ranges for fluctuating these parameters. The authors found that the most influential factors on NPV include the price of the extract, the interest rate, the dynamic time of SFE, and the project lifetime. Moreover, in this study tried to determine rhe upper and lower bounds for technical and economic factors, utilizing pertinent literature data. 
	
	\citet{Poletto1996} provided a general dimensionless model for the supercritical vegetable and essential oils extraction process and applied a sensitivity analysis. They found that a dimensionless partition coefficient and a dimensionless characteristic time are the most influential parameters of the extraction process. The sensitivity calculations were performed by varying the parameters and analysing the model response.
	
\end{document}