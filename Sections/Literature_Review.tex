\documentclass[../Article_Sensitivity_Analsysis.tex]{subfiles}
\graphicspath{{\subfix{../Figures/}}}
\begin{document}
	
	This study aims to analyse the influence of changes in operating conditions on the process model described in {\color{red}article 1}. The relation between input and output is obtained by applying a sensitivity analysis, which examines the impact of model parameters or controls on the model output. The result of sensitivity analysis can be used to allocate the source of uncertainty in the system, perform the model reduction or search for errors in the model (by identifying unexpected relationships between inputs and outputs). There are many sensitivity analysis methods, which include but are not limited to those listed below:
	
	\begin{itemize}
		\item One-at-a-time method
		\item Derivative-based local methods
		\item Variance-based methods
	\end{itemize}
	
	In the literature, many different supercritical extraction models were analysed using different sensitivity analysis techniques. \citet{Fiori_2007}, performed the sensitivity calculations by varying the parameters within their confidence interval and observing how the model results changed. Their sensitivity analysis revealed that changes in particle diameter and the internal mass transfer coefficient have the strongest influence on extraction in the second phase of the process (diffusion-control regime).% The effect of changing some operative conditions was also investigated, underlining how the solvent flow rate and the seed milling affect the extraction process.
	
	\citet{Santos2000}, in their work, considered the model of \citet{Sovova1994} for the semi-continuous isothermal and isobaric extraction processes using carbon dioxide as a solvent. The parametric sensitivity analysis was carried out using a factorial design in two levels. The model parameters were disturbed by 10\%, and their main effects were analysed and strategies for high-performance operation are proposed. The authors performed the sensitivity analysis with respect to the superficial velocity, particle diameter, initial concentration of solute in the solid phase and the concentration of solute in the fluid phase at the inlet to the extractor.
	
	\citet{Hatami2024} performed the one-factor-at-a-time sensitivity analysis to assess the response of net present value concerning variations in both technical and economic variables. Their study is divided into two parts. In the first part, the sensitivity analysis is carried out for an extraction plant with a constant extractor volume of 300 L. The focus is on examining how net present value is influenced by changes in individual technical and economic parameters while keeping the extractor volume fixed. In the second part, the investigation revolves around the effects of altering the extractor volume (from 1 to 600 L) on the project's overall profitability. Their findings show that the most influential factors on net present value include the price of the extract, the interest rate, the dynamic time, and the project lifetime.
	
	%\citet{Poletto1996} provided a general dimensionless model for the supercritical extraction process of vegetable and essential oils. The sensitivity calculations were performed by varying the parameters and analysing the model response. The authors found that a dimensionless partition coefficient and a dimensionless characteristic time appeared as the most important parameters of the extraction process. 
	
\end{document}