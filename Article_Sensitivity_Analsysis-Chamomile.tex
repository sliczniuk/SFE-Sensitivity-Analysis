% ---------------------------------------------------------------
% Preamble
% ---------------------------------------------------------------
%\documentclass[a4paper,fleqn,longmktitle]{cas-sc}
\documentclass[a4paper,fleqn]{cas-dc}
%\documentclass[a4paper]{cas-dc}
%\documentclass[a4paper]{cas-sc}
% ---------------------------------------------------------------
% Make margins bigger to fit annotations. Use 1, 2 and 3. TO be removed later
%\paperwidth=\dimexpr \paperwidth + 6cm\relax
%\oddsidemargin=\dimexpr\oddsidemargin + 3cm\relax
%\evensidemargin=\dimexpr\evensidemargin + 3cm\relax
%\marginparwidth=\dimexpr \marginparwidth + 3cm\relax
% -------------------------------------------------------------------- 
% Packages
% --------------------------------------------------------------------
% Figure packages
\usepackage{graphicx,float}
\usepackage{adjustbox}
% Text, input, formatting, and language-related packages
\usepackage[T1]{fontenc}
\usepackage{subcaption}

\usepackage{csvsimple}

% TODO package
\usepackage[bordercolor=gray!20,backgroundcolor=blue!10,linecolor=black,textsize=footnotesize,textwidth=1in]{todonotes}
\setlength{\marginparwidth}{1in}
% \usepackage[utf8]{inputenc}
% \usepackage[nomath]{lmodern}

% Margin and formatting specifications
%\usepackage[authoryear]{natbib}
\usepackage[sort]{natbib}
\setcitestyle{square,numbers}

 %\bibliographystyle{cas-model2-names}

\usepackage{setspace}
\usepackage{subfiles} % Best loaded last in the preamble

% \usepackage[authoryear,longnamesfirst]{natbib}

% Math packages
\usepackage{amsmath, amsthm, amssymb, amsfonts, bm, nccmath, mathdots, mathtools, bigints, ulem}

\usepackage{tikz}
\usepackage{pgfplots}
\usetikzlibrary{shapes.geometric,angles,quotes,calc}

\usepackage{placeins}

\usepackage[final]{pdfpages}

% --------------------------------------------------------------------
% Packages Configurations
\usepackage{enumitem}
% --------------------------------------------------------------------
% (General) General configurations and fixes
\AtBeginDocument{\setlength{\FullWidth}{\textwidth}}	% Solves els-cas caption positioning issue
\setlength{\parindent}{20pt}
%\doublespacing
% --------------------------------------------------------------------
% Other Definitions
% --------------------------------------------------------------------
\graphicspath{{Figures/}}
% --------------------------------------------------------------------
% Environments
% --------------------------------------------------------------------
% ...

% --------------------------------------------------------------------
% Commands
% --------------------------------------------------------------------

% ==============================================================
% ========================== DOCUMENT ==========================
% ==============================================================
\begin{document} 
%  --------------------------------------------------------------------

% ===================================================
% METADATA
% ===================================================
\title[mode=title]{Sensitivity Analysis}                      
\shorttitle{Sensitivity Analysis}

\shortauthors{OS, PO}

\author[1]{Oliwer Sliczniuk}[orcid=0000-0003-2593-5956]
\ead{oliwer.sliczniuk@aalto.fi}
\cormark[1]
\credit{a}

\author[1]{Pekka Oinas}[orcid=0000-0002-0183-5558]
\credit{b}

%\author[1]{Francesco Corona}[orcid=0000-0002-3615-1359]
%\credit{c}

\address[1]{Aalto University, School of Chemical Engineering, Espoo, 02150, Finland}
%\address[2]{2}

\cortext[cor1]{Corresponding author}

% ===================================================
% ABSTRACT
% ===================================================
\begin{abstract}
This study investigates the process of caraway oil extraction from chamomile flowers. A parameter-distributed model consisting of a set of partial differential equations is used to describe the governing mass transfer phenomena in a solid-fluid environment under supercritical conditions using carbon dioxide as a solvent. The concept of quasi-one-dimensional flow is applied to reduce the number of spatial dimensions. The flow is assumed to be uniform across any cross-section, although the area available for the fluid phase can vary along the extractor. The physical properties of the solvent are estimated from the Peng-Robinson equation of state. A set of laboratory experiments was performed under multiple constant operating conditions: $30 - 40^\circ C$, $100 - 200$ bar, and $3.33-6.67 \cdot 10^{-5}$ kg/s. Different sensitivity analysis methods can be applied to assess the robustness of the model parameters and their influence on the process model. The local sensitivity analysis investigates the influence of infinitely small changes in the model parameters and the controls on model output. This work focuses on analysing the effect of pressure on the model state space and the extraction yield.

\end{abstract}

\begin{keywords}
Supercritical extraction \sep Sensitivity analysis \sep Mathematical modelling
\end{keywords}

% ===================================================
% TITLE
% ===================================================
\maketitle

% ===================================================
% Section: Introduction
% ===================================================

\section{Introduction}

\subfile{Sections/introduction_imp}

\subfile{Sections/Literature_Review}

% ===================================================
% Section: Main
% ===================================================

\subfile{Sections/Model}

\subsection{Local sensitivity analysis} \label{CH: Sensitivity_Analysis}
\subfile{Sections/Sensitivity_Analysis}

% ===================================================
% Section: Summary
% ===================================================

\section{Results}
\subfile{Sections/Results_Sensitivity}

\section{Conclusions} \label{CH: Conclusion}

Sensitivity analysis is a tool used to understand how model parameters affect its output. The presented formulation involves derivative-based local sensitivity analysis of the model solution with respect to selected parameters and controls. This work implemented the automatic differentiation to derive the sensitivity equations. The local sensitivity analysis techniques consider only a small region of parameter space, and the conclusions derived from such an analysis are limited to local conditions. In the case of dynamical systems, local sensitivity analysis provides a time series describing how that dependency evolves. 

The local sensitivity analysis can be performed with respect to any model parameter, but this work focuses on the effect of pressure increase. At selected operating conditions ($35^\circ C$, 150 bar and 5 $\cdot 10^{-5}$ kg/s), the pressure increase enhances the mass transfer, which leads to faster loss of solute from particles and consequently to negative sensitivities in the solid phase. Analogously, the sensitivities in the fluid phase are characterised by positive deviations, which indicates that more solute is transported to the fluid phase. As a result, the extraction yield is improved and characterised by positive sensitivities. Local sensitivity analysis can provide valuable information by identifying which parameters are influential and how these influence changes over time. The sensitivity analysis results can be utilised for process modelling, experiment design, or model reduction.

% ===================================================
% Bibliography
% ===================================================
%% Loading bibliography style file
%\clearpage
%\bibliographystyle{model1-num-names}
\bibliographystyle{unsrtnat}
\bibliography{mybibfile}

%\clearpage \appendix \label{appendix}
%\section{Appendix} 
%\subfile{Sections/Qubic_EOS} \label{CH: EOS}
%\subsection{Cardano's Formula} \label{CH: Cardano}
%\subfile{Sections/Cardano}

\end{document}