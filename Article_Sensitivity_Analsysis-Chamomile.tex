% ---------------------------------------------------------------
% Preamble
% ---------------------------------------------------------------
%\documentclass[a4paper,fleqn,longmktitle]{cas-sc}
\documentclass[a4paper,fleqn]{cas-dc}
%\documentclass[a4paper]{cas-dc}
%\documentclass[a4paper]{cas-sc}
% ---------------------------------------------------------------
% Make margins bigger to fit annotations. Use 1, 2 and 3. TO be removed later
%\paperwidth=\dimexpr \paperwidth + 6cm\relax
%\oddsidemargin=\dimexpr\oddsidemargin + 3cm\relax
%\evensidemargin=\dimexpr\evensidemargin + 3cm\relax
%\marginparwidth=\dimexpr \marginparwidth + 3cm\relax
% -------------------------------------------------------------------- 
% Packages
% --------------------------------------------------------------------
% Figure packages
\usepackage{graphicx,float}
\usepackage{adjustbox}
% Text, input, formatting, and language-related packages
\usepackage[T1]{fontenc}
\usepackage{subcaption}

\usepackage{nomencl}
\makenomenclature

\usepackage{etoolbox}
\renewcommand\nomgroup[1]{%
	\item[\bfseries
	\ifstrequal{#1}{A}{Latin symbols}{%
		\ifstrequal{#1}{B}{Greek symbols}{%
			\ifstrequal{#1}{C}{Abberivations}{
	}}}%
	]}

\usepackage{csvsimple}

% TODO package
\usepackage[bordercolor=gray!20,backgroundcolor=blue!10,linecolor=black,textsize=footnotesize,textwidth=1in]{todonotes}
\setlength{\marginparwidth}{1in}
% \usepackage[utf8]{inputenc}
% \usepackage[nomath]{lmodern}

% Margin and formatting specifications
%\usepackage[authoryear]{natbib}
\usepackage[sort]{natbib}
\setcitestyle{square,numbers}

 %\bibliographystyle{cas-model2-names}

\usepackage{setspace}
\usepackage{subfiles} % Best loaded last in the preamble

% \usepackage[authoryear,longnamesfirst]{natbib}

% Math packages
\usepackage{amsmath, amsthm, amssymb, amsfonts, bm, nccmath, mathdots, mathtools, bigints, ulem}

\usepackage{tikz}
\usepackage{pgfplots}
\usetikzlibrary{shapes.geometric,angles,quotes,calc}

\usepackage{placeins}

\usepackage[final]{pdfpages}

% --------------------------------------------------------------------
% Packages Configurations
\usepackage{enumitem}
% --------------------------------------------------------------------
% (General) General configurations and fixes
\AtBeginDocument{\setlength{\FullWidth}{\textwidth}}	% Solves els-cas caption positioning issue
\setlength{\parindent}{20pt}
%\doublespacing
% --------------------------------------------------------------------
% Other Definitions
% --------------------------------------------------------------------
\graphicspath{{Figures/}}
% --------------------------------------------------------------------
% Environments
% --------------------------------------------------------------------
% ...

% --------------------------------------------------------------------
% Commands
% --------------------------------------------------------------------

% ==============================================================
% ========================== DOCUMENT ==========================
% ==============================================================
\begin{document} 
%  --------------------------------------------------------------------

% ===================================================
% METADATA
% ===================================================
\title[mode=title]{Sensitivity Analysis}                      
\shorttitle{Sensitivity Analysis}

\shortauthors{OS, PO}

\author[1]{Oliwer Sliczniuk}[orcid=0000-0003-2593-5956]
\ead{oliwer.sliczniuk@aalto.fi}
\cormark[1]
\credit{a}

\author[1]{Pekka Oinas}[orcid=0000-0002-0183-5558]
\credit{b}

%\author[1]{Francesco Corona}[orcid=0000-0002-3615-1359]
%\credit{c}

\address[1]{Aalto University, School of Chemical Engineering, Espoo, 02150, Finland}
%\address[2]{2}

\cortext[cor1]{Corresponding author}

% ===================================================
% ABSTRACT
% ===================================================
\begin{abstract}
This study investigates the process of chamomile oil extraction from chamomile flowers. A parameter-distributed model, consisting of a set of partial differential equations, is used to describe the governing mass transfer phenomena between solid and fluid phases under supercritical conditions using carbon dioxide as the solvent. The concept of quasi-one-dimensional flow is applied to reduce the number of spatial dimensions. The flow is assumed to be uniform across any cross-section, although the area available for the fluid phase can vary along the extractor. The physical properties of the solvent are estimated using the Peng-Robinson equation of state. Laboratory experiments were conducted under various, but constant operating conditions: $30 - 40^\circ C$, $100 - 200$ bar, and $3.33-6.67 \cdot 10^{-5}$ kg/s. Different sensitivity analysis methods can be applied to assess the robustness of the model parameters and their influence on the process model. Local sensitivity analysis investigates the impact of infinitesimally small changes in model parameters and controls on model output. This study focuses on analysing the effect of pressure on the model state space and extraction yield.

\end{abstract}

\begin{keywords}
Supercritical extraction \sep Sensitivity analysis \sep Mathematical modelling
\end{keywords}

% ===================================================
% TITLE
% ===================================================
\maketitle

% ===================================================
% Section: Introduction
% ===================================================

\section{Introduction}

\subfile{Sections/introduction_imp}

\subfile{Sections/Literature_Review}

% ===================================================
% Section: Main
% ===================================================

\subfile{Sections/Model}

\subsection{Local sensitivity analysis} \label{CH: Sensitivity_Analysis}
\subfile{Sections/Sensitivity_Analysis}

% ===================================================
% Section: Summary
% ===================================================

\section{Results}
\subfile{Sections/Results_Sensitivity}

\section{Conclusions} \label{CH: Conclusion}

Sensitivity analysis is a tool used to understand how model parameters affect its output. The presented formulation involves derivative-based local sensitivity analysis of the model solution with respect to selected parameters and controls. This work implemented the automatic differentiation to derive the sensitivity equations. The local sensitivity analysis techniques consider only a small region of parameter space, and the conclusions derived from such an analysis are limited to local conditions. In the case of dynamical systems, the outcome of local sensitivity analysis quantifies how the state of the system or its output varies in time with respect to small perturbations in parameters.

The local sensitivity analysis can be performed with respect to any model parameter, but this work focuses on the effect of pressure increase. At selected operating conditions ($35^\circ C$ and $5\cdot 10^{-5}$ kg/s), the pressure increase enhances the mass transfer, which leads to faster loss of solute from particles and consequently to negative sensitivities in the solid phase. Analogously, the sensitivities in the fluid phase are characterised by positive deviations, which indicates that more solute is transported to the fluid phase. As a result, the extraction yield is improved and characterised by positive sensitivities. As the results of the local sensitivity analysis depends on the operating point, the analysis was repeated at different nominal pressures. It can be observed that system responses at low-pressures are stronger than at high-pressures. This behaviour can be explained by rapid changes of the solvent properties close to the critical point.

Local sensitivity analysis results provide valuable information by identifying which parameters are influential and how these influence changes over time. One application of the sensitivity analysis is to spot discrepancies between model predictions and actual experimental results. Identifying unexpected system responses prompts further investigation and refinement of the process model. Sensitivity analysis results can be used to design future experiments by indicating which parameters should be varied and monitored. Moreover, the parameters which are characterized by low sensitivity can be subject of model reduction. 

% ===================================================
% Bibliography
% ===================================================
%% Loading bibliography style file
%\clearpage
\newpage
%\bibliographystyle{model1-num-names}
\bibliographystyle{unsrtnat}
\bibliography{mybibfile}

%\clearpage \appendix \label{appendix}
%\section{Appendix} 
%\subfile{Sections/Qubic_EOS} \label{CH: EOS}
%\subsection{Cardano's Formula} \label{CH: Cardano}
%\subfile{Sections/Cardano}

\newpage

\nomenclature[A]{\(P\)}{Pressure}
\nomenclature[A]{\(T\)}{Temperature}
\nomenclature[A]{\(T^{in}\)}{Inlet temperature}
\nomenclature[A]{\(T^{out}\)}{Outlet temperature}
\nomenclature[A]{\(F\)}{Mass flow rate}
\nomenclature[A]{\(u\)}{Superfical velocity}
\nomenclature[A]{\(v\)}{Linear velocity}
\nomenclature[A]{\(z\)}{Spatial direction}
\nomenclature[A]{\(e\)}{Internal energy}
\nomenclature[A]{\(h\)}{Enthalpy}
\nomenclature[A]{\(t\)}{Time}
\nomenclature[A]{\(t_f\)}{Total extraction time}
\nomenclature[A]{\(t_0\)}{Inital extraction time}
\nomenclature[A]{\(A\)}{Total cross section of the bed}
\nomenclature[A]{\(A_f\)}{Cross section of the bed accupied by the fluid}
\nomenclature[A]{\(c_f\)}{Concentration of solute in fluid phase}
\nomenclature[A]{\(c_{f0}\)}{Inital concentration of solute in fluid phase}
\nomenclature[A]{\(c_f^*\)}{Concentration of solute at the solid-fluid interface}
\nomenclature[A]{\(c_s\)}{Concentration of solute in solid phase}
\nomenclature[A]{\(c_{s0}\)}{Inital concentration of solute in solid phase}
\nomenclature[A]{\(c_s^*\)}{Concentration of solute at the solid-fluid interface}
\nomenclature[A]{\(c_p\)}{Concentration of solute in the core of a pore}
\nomenclature[A]{\(c_{pf}\)}{Concentration of solute in the pore opening}
\nomenclature[A]{\(D_e^M\)}{Axial diffusion coefficeint}
\nomenclature[A]{\(r_e\)}{Mass transfer kinetic term}
\nomenclature[A]{\(L\)}{Length of fixed bed}
\nomenclature[A]{\(S\)}{Sensitivity equations}
\nomenclature[A]{\(\textbf{G}\)}{Augumented system}
\nomenclature[A]{\(\dot{S}\)}{Time derivative of sensitivity equations}
\nomenclature[A]{\(\bar{S}\)}{Sensitivity matrix}
\nomenclature[A]{\(\bar{J}_x\)}{Jacobian matrix with respect to the state space}
\nomenclature[A]{\(\bar{J}_\Theta\)}{Jacobian matrix with respect to the paramters}
%\nomenclature[A]{\(\mathcal{L}\)}{Log-likelihood function}
\nomenclature[A]{\(l\)}{Characterisitic diemsinon of particles}
\nomenclature[A]{\(D_i\)}{Interanl diffusion coefficient}
\nomenclature[A]{\(D_i^R\)}{Reference value of interanl diffusion coefficient}
\nomenclature[A]{\(r\)}{Particle radius}
\nomenclature[A]{\(k_p\)}{Volumetric partition coefficient}
\nomenclature[A]{\(k_m\)}{Mass partition coefficient}
\nomenclature[A]{\(y\)}{Extraction yield}
\nomenclature[A]{\(Y\)}{Yield measurment}
\nomenclature[A]{\(x\)}{State vector}
\nomenclature[A]{\(G\)}{Vector of discretized differential equations}
\nomenclature[A]{\(p\)}{Probaility disribution model}
\nomenclature[A]{\(R_e\)}{Reynolds number}

\nomenclature[B]{\(\rho_f\)}{Fluid density}
\nomenclature[B]{\(\Phi\)}{Bed porosity}
\nomenclature[B]{\(\rho_s\)}{Bulk density of solid}
\nomenclature[B]{\(\mu\)}{Sphericity coefficient}
\nomenclature[B]{\(\gamma\)}{Decaying function}
\nomenclature[B]{\(\Upsilon\)}{Decay coefficient}
\nomenclature[B]{\(\Theta\)}{Paramter space}
\nomenclature[B]{\(\theta\)}{vector of unknown parameters}
\nomenclature[B]{\(\epsilon\)}{Unobervable error}
\nomenclature[B]{\(\sigma\)}{Standard deviation}

\nomenclature[C]{BIC}{Broke-and-Intact Cell model}
%\nomenclature[C]{MLE}{Maximum Likelihood Estimation}
\nomenclature[C]{SFE}{Supercritical Fluid Extraction}
\nomenclature[C]{HBD}{Hot Ball Diffusion}
\nomenclature[C]{SC}{Shrinking Core}

\printnomenclature

\end{document}