% ---------------------------------------------------------------
% Preamble
% ---------------------------------------------------------------
%\documentclass[a4paper,fleqn,longmktitle]{cas-sc}
\documentclass[a4paper,fleqn]{cas-dc}
%\documentclass[a4paper]{cas-dc}
%\documentclass[a4paper]{cas-sc}
% ---------------------------------------------------------------
% Make margins bigger to fit annotations. Use 1, 2 and 3. TO be removed later
%\paperwidth=\dimexpr \paperwidth + 6cm\relax
%\oddsidemargin=\dimexpr\oddsidemargin + 3cm\relax
%\evensidemargin=\dimexpr\evensidemargin + 3cm\relax
%\marginparwidth=\dimexpr \marginparwidth + 3cm\relax
% -------------------------------------------------------------------- 
% Packages
% --------------------------------------------------------------------
% Figure packages
\usepackage{graphicx,float}
\usepackage{adjustbox}
% Text, input, formatting, and language-related packages
\usepackage[T1]{fontenc}
\usepackage{subcaption}

\usepackage{csvsimple}

% TODO package
\usepackage[bordercolor=gray!20,backgroundcolor=blue!10,linecolor=black,textsize=footnotesize,textwidth=1in]{todonotes}
\setlength{\marginparwidth}{1in}
% \usepackage[utf8]{inputenc}
% \usepackage[nomath]{lmodern}

% Margin and formatting specifications
%\usepackage[authoryear]{natbib}
\usepackage[sort]{natbib}
\setcitestyle{square,numbers}

 %\bibliographystyle{cas-model2-names}

\usepackage{setspace}
\usepackage{subfiles} % Best loaded last in the preamble

% \usepackage[authoryear,longnamesfirst]{natbib}

% Math packages
\usepackage{amsmath, amsthm, amssymb, amsfonts, bm, nccmath, mathdots, mathtools, bigints, ulem}

\usepackage{tikz}
\usepackage{pgfplots}
\usetikzlibrary{shapes.geometric,angles,quotes,calc}

\usepackage{placeins}

\usepackage[final]{pdfpages}

% --------------------------------------------------------------------
% Packages Configurations
\usepackage{enumitem}
% --------------------------------------------------------------------
% (General) General configurations and fixes
\AtBeginDocument{\setlength{\FullWidth}{\textwidth}}	% Solves els-cas caption positioning issue
\setlength{\parindent}{20pt}
%\doublespacing
% --------------------------------------------------------------------
% Other Definitions
% --------------------------------------------------------------------
\graphicspath{{Figures/}}
% --------------------------------------------------------------------
% Environments
% --------------------------------------------------------------------
% ...

% --------------------------------------------------------------------
% Commands
% --------------------------------------------------------------------

% ==============================================================
% ========================== DOCUMENT ==========================
% ==============================================================
\begin{document} 
%  --------------------------------------------------------------------

% ===================================================
% METADATA
% ===================================================
\title[mode=title]{Sensitivity Analysis}                      
\shorttitle{Sensitivity Analysis}

\shortauthors{OS, PO}

\author[1]{Oliwer Sliczniuk}[orcid=0000-0003-2593-5956]
\ead{oliwer.sliczniuk@aalto.fi}
\cormark[1]
\credit{a}

\author[1]{Pekka Oinas}[orcid=0000-0002-0183-5558]
\credit{b}

%\author[1]{Francesco Corona}[orcid=0000-0002-3615-1359]
%\credit{c}

\address[1]{Aalto University, School of Chemical Engineering, Espoo, 02150, Finland}
%\address[2]{2}

\cortext[cor1]{Corresponding author}

% ===================================================
% ABSTRACT
% ===================================================
\begin{abstract}
This study aimed to investigate the supercritical extraction process of caraway oil from chamomile flowers. The distributed-parameter model describes the fluid-solid extraction process. The concept of quasi-one-dimensional flow is applied to reduce the number of spatial dimensions. The flow is assumed to be uniform across any cross-section, although the area available for the fluid phase can vary along the extractor. The physical properties of the solvent are estimated from the Peng-Robinson equation of state. A set of laboratory experiments was performed under multiple constant operating conditions: $30 - 40^\circ C$, $100 - 200$ bar, and $3.33-6.67 \cdot 10^{-5}$ kg/s. Sensitivity analyses play a crucial role in assessing the robustness of the findings or conclusions based on mathematical model. The local sensitivity analysis investigates the influence of infinitely small changes in the inlet temperature, pressure, and flow rate on the extraction yield.

\end{abstract}

\begin{keywords}
Supercritical extraction \sep Sensitivity analysis \sep Mathematical modelling
\end{keywords}

% ===================================================
% TITLE
% ===================================================
\maketitle

% ===================================================
% Section: Introduction
% ===================================================

\section{Introduction}

\subfile{Sections/introduction_imp}

\subfile{Sections/Literature_Review}

% ===================================================
% Section: Main
% ===================================================

\subfile{Sections/Model}

\subsection{Local sensitivity analysis} \label{CH: Sensitivity_Analysis}
\subfile{Sections/Sensitivity_Analysis}

% ===================================================
% Section: Summary
% ===================================================

\section{Results}
\subfile{Sections/Results_Sensitivity}

\section{Conclusions} \label{CH: Conclusion}

Sensitivity analysis is a tool to understand how parameters affect a model's output. In the case of dynamical systems, local sensitivity analysis provides a time series describing how that dependency evolves. The presented formulation involves derivative-based local sensitivity analysis of the model solution with respect to selected parameters and controls. The local sensitivity analysis techniques consider only a small region of parameter space, and the conclusions derived from such an analysis are limited to local conditions unless the discussed system is a linear model. The sensitivity equations can be obtained in various ways. This work implemented the automatic differentiation technique to derive the sensitivity equations. The effect of pressure increase on the outcome of the supercritical extraction model is analysed in this work. At given operating conditions ($35^\circ C$, 150 bar and 5 $\cdot 10^{-5}$ kg/s), the step change of pressure enhance the mass transfer, which leads to faster loss of solute from solid particles and consequently to negative sensitivities in the solid phase.  Analogously, the sensitivities in the fluid phase are characterized by positive deviations, which indicates that more solute is transported to the fluid phase. As a result, the extraction yield is improved and characterised by positive sensitivities as well. Local sensitivity analysis can provide valuable information about process modelling, experiment design, or model reduction by identifying which parameters are influential and how these influence changes over time.

% ===================================================
% Bibliography
% ===================================================
%% Loading bibliography style file
%\clearpage
%\bibliographystyle{model1-num-names}
\bibliographystyle{unsrtnat}
\bibliography{mybibfile}

%\clearpage \appendix \label{appendix}
%\section{Appendix} 
%\subfile{Sections/Qubic_EOS} \label{CH: EOS}
%\subsection{Cardano's Formula} \label{CH: Cardano}
%\subfile{Sections/Cardano}

\end{document}